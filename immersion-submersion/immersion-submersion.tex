\documentclass{article}
\usepackage{ctex}
\usepackage{amsmath}
\usepackage{amssymb}
\usepackage{listings}
\usepackage{amsthm}
\usepackage{booktabs}
\setlength{\parindent}{0pt}
\newtheorem{definition}{定义}[section]
\newtheorem{thm}{定理}[section]
\newtheorem{lemma}{引理}[section]
\newtheorem{example}{例}
\newtheorem{note}{注}
\newtheorem{solution}{解}
\newtheorem{corollary}{推论}
\newtheorem{proposition}{性质}[section]
\lstset{language=Matlab}
\lstset{breaklines}
\lstset{extendedchars=false} %取消首行缩进
\begin{document}
\begin{center}
    \LARGE
    \textbf{浸入、淹没、嵌入及子流形}\\
    \vspace{0.2em}
    \large
\end{center}
\section{浸入}
\section{子流形}
\subsection{正则子流形的局部结构}
正则子流形的结构非常简单
\begin{proposition}
    每个浸入子流形的局部结构是一个坐标片
\end{proposition}
下面给出两个证明流形的子集是嵌入子流形的充分条件
\begin{thm}\label{submanifold-1}[淹没定理]
    设$f:M^{m} \to N^{n}$为光滑映射,$q \in f(M)$。如果$f$在$A=f^{-1}(q)$上的每一点为淹没,则
    $A$是$M$的$m-n$维闭正则子流形
\end{thm}
\begin{thm}[子浸入定理]
    设$f:M^{m} \to N^{n}$为光滑映射,且${\rm rank} f \equiv r$。对于$q \in f(M)$,
    $A=f^{-1}(q)$是$M$的$m-r$维闭正则子流形
\end{thm}
\section{Stiefel流形}
\begin{definition}[${\rm Stiefel}$流形]
    设${\rm St}(p,n) \ (p \leq n)$表示所有$n \times p$的正交矩阵的集合,即
    $$
    {\rm St}(p,n):=\{X \in \mathbb{R}^{n \times p}:X^TX=I_{p}\}
    $$
    其中$I_{p}$表示$p \times p$的单位矩阵。我们称${\rm St}(p,n)$为${\rm Stiefel}$流形
\end{definition}
\begin{proposition}
    ${\rm Stiefel}$流形是$\mathbb{R}^{n \times p}$的嵌入子流形
\end{proposition}
\begin{proof}
    考虑映射$f:\mathbb{R}^{n \times p} \to S_{p \times p}:X \mapsto X^TX-I_{p}$,其中$S_{p \times p}$表示
    所有$p \times p$的对称矩阵。显然,${\rm St}(p,n)=f^{-1}(0_{p})\subset \mathbb{R}^{n \times p}$。下证$f$在${\rm St}(p,n)$中的每一点$X$都是浸入,即证明
    ${\rm rank}(f)={\rm dim}(S_{p \times p})$。因为$f$的定义域和值域都为线性空间,因此计算$f$的秩不依赖坐标图。即
    ${\rm rank}(f)=J(f)$,对于$f$来说,求${\rm jacobian}$矩阵就是对$f$求导,$f$在$X$点的导数为
    $$
    \begin{aligned}
        Df(X)[Z]&=\lim_{\alpha \to 0^+}\frac{f(X+\alpha Z)-f(X)}{\alpha}\\
        &=\lim_{\alpha \to 0^+}\frac{(X+\alpha Z)^T(X+\alpha Z)-X^TX}{\alpha}\\
        &=\lim_{\alpha \to 0^+}X^TZ+Z^TX+\alpha Z^TZ\\
        &=X^TZ+Z^TX
    \end{aligned}
    $$
    对任意$X \in {\rm St}(p,n)$,选取$\frac{1}{2}X\hat{Z}$作为方向,其中$\hat{Z} \in S_{p \times p}$,则
    $$
    Df(X)\left[\frac{1}{2}X\hat{Z}\right]=\hat{Z} 
    $$
    因为$X^TX=I_{p},{\hat{Z}}^T=\hat{Z}$。因此${\rm dim}(f)={\rm dim}(S_{p \times p})$,从而$f$是浸入。
    由定理\ref{submanifold-1}可得,${\rm St}(p,n)$是$\mathbb{R}^{n \times p}$的嵌入子流形
\end{proof}
\end{document}